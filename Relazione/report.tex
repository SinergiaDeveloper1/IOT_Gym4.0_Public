%%%%%%%%%%%%%%%%%%%%%%%%%%%%%%%%%%%%%%%%%
% Stylish Article
% LaTeX Template
% Version 2.2 (2020-10-22)
%
% This template has been downloaded from:
% http://www.LaTeXTemplates.com
%
% Original author:
% Mathias Legrand (legrand.mathias@gmail.com) 
% With extensive modifications by:
% Vel (vel@latextemplates.com)
%
% License:
% CC BY-NC-SA 3.0 (http://creativecommons.org/licenses/by-nc-sa/3.0/)
%
%%%%%%%%%%%%%%%%%%%%%%%%%%%%%%%%%%%%%%%%%

%----------------------------------------------------------------------------------------
%	PACKAGES AND OTHER DOCUMENT CONFIGURATIONS
%----------------------------------------------------------------------------------------

\documentclass[fleqn,10pt]{SelfArx} % Document font size and equations flushed left

%\usepackage[english]{babel} % Specify a different language here - english by default
\usepackage[italian]{babel} % Specify a different language here - english by default
\usepackage{attachfile}
\usepackage{biblatex}
%----------------------------------------------------------------------------------------
%	COLUMNS
%----------------------------------------------------------------------------------------

\setlength{\columnsep}{0.55cm} % Distance between the two columns of text
\setlength{\fboxrule}{0.75pt} % Width of the border around the abstract

%----------------------------------------------------------------------------------------
%	COLORS
%----------------------------------------------------------------------------------------

\definecolor{color1}{RGB}{0,0,90} % Color of the article title and sections
\definecolor{color2}{RGB}{0,20,20} % Color of the boxes behind the abstract and headings

%----------------------------------------------------------------------------------------
%	HYPERLINKS
%----------------------------------------------------------------------------------------

\usepackage{hyperref} % Required for hyperlinks

\hypersetup{
	hidelinks,
	colorlinks,
	breaklinks=true,
	urlcolor=color2,
	citecolor=color1,
	linkcolor=color1,
	bookmarksopen=false,
	pdftitle={Title},
	pdfauthor={Author},
}

%----------------------------------------------------------------------------------------
%	ARTICLE INFORMATION
%----------------------------------------------------------------------------------------

\JournalInfo{Programmazione per l'IoT - Laurea Magistrale in Informatica Applicata - DiSPeA - Università degli Studi di Urbino Carlo Bo} % Journal information
\Archive{Data di pubblicazione xx/xx/xxxx - DOI: xxxx/xxxxxxx} % Informazioni che verranno inserite dal Docente in fase di pubblicazione

\PaperTitle{4.0 Gym Prototype} % Article title

\Authors{Luca Cinti\textsuperscript{1}*, Emanuele Lattanzi\textsuperscript{2}} % Il docente Emanuele Lattanzi figura come autore
% al fine di poter gestire la procedura di submission sul repository pubblico (nell'affiliazione viene chiarito il ruolo)
\affiliation{\textsuperscript{1}\textit{Laurea Magistrale in Informatica Applicata, Università degli Studi di Urbino Carlo Bo, Urbino, Italia}} % Author affiliation
\affiliation{\textsuperscript{2}\textit{Docente di Programmazione per l'Internet of Things, Università degli Studi di Urbino Carlo Bo, Urbino, Italia}} % Author affiliation
\affiliation{*\textbf{Corresponding author}: l.cinti@campus.uniurb.it} % Corresponding author

\Keywords{IOT --- Arduino --- ESP32 --- MQTT --- Training Metrics} % Keywords - if you don't want any simply remove all the text between the curly brackets
\newcommand{\keywordname}{Keywords} % Defines the keywords heading name

%----------------------------------------------------------------------------------------
%	ABSTRACT
%----------------------------------------------------------------------------------------

\Abstract{
	In questo elaborato abbiamo cercato di predisporre un'architettura per la raccolta dei dati
	in una piccola home gym, innanzitutto per verificare la fattibilità nel reperimento di alcune misurazioni, 
	per poi passare all'analisi dei dati raccolti, con l'obiettivo di estrapolare delle metriche significative 
	sia ai fini dell'allenamento, che dello studio delle caratteristiche termiche dell'ambiente.
}

%----------------------------------------------------------------------------------------

\addbibresource{sample.bib}

\begin{document}

\maketitle % Output the title and abstract box

%\tableofcontents % Output the contents section

\thispagestyle{empty} % Removes page numbering from the first page

%----------------------------------------------------------------------------------------
%	ARTICLE CONTENTS
%----------------------------------------------------------------------------------------

\section*{Introduzione} % The \section*{} command stops section numbering

L'obiettivo preliminare di questo elaborato era verificare la fattibilità dell'allestimento di un sistema di 
raccolta dati, in una piccola home gym, costruendo l'infrastruttura per far comunicare 
attraverso diversi protocolli una rete di sensori.\\

Tra le più disparate metriche di possibile interesse ai fini dello studio di una palestra, 
sono state scelte due categorie di dati: da un lato i quelli relativi all'ambiente di allenamento, 
dall'altro i dati sull'esecuzione degli esercizi.

Per il progetto corrente sono state dunque prese in analisi le metriche seguenti:

\begin{itemize}[noitemsep] % [noitemsep] removes whitespace between the items for a compact look
	\item temperatura e umidità della stanza, prima durante e dopo l'allenamento
	\item variazione di eCO\textsubscript{2} e eTVOC (equivalent Total Volatile Organic Compounds)
	\item accelerazione del bilanciere durante l'esecuzione di un esercizio campione
\end{itemize}

\section{Descrizione dell'ambiente}

Funzionale alla comprensione di questo elaborato, è la descrizione dell'ambiente in cui è stata allestita 
la palestra: si tratta di una stanza appartenente ad un vecchio immobile disabitato, appositamente 
ristrutturata, le cui misure sono 5.05 x 4.95 metri, per un'altezza di 3.86 metri. \\

L'ambiente presenta una finestra, una porta finestra e un'arcata di ingresso sulla quale sono state applicate 
due tende, fissate con velcro ai lati del muro, come isolante dal resto del locale, in quanto non 
sono presenti sistemi di riscaldamento centralizzato. Possiamo vederne una panoramica nell'immagine che segue 
(Figura 2). \\

Il primo punto di analisi è dunque diretta conseguenza di quanto appena detto: analizzare le performance 
termiche dell'ambiente, per trovare il miglior metodo di riscaldamento durante i mesi invernali.\\
Collegato a questo vi è il secondo punto in analisi: riscaldando un ambiente di circa 96 metri cubici, quanto 
più possibile isolato dal resto del locale, abbiamo ritenuto importante monitorare la qualità dell'aria 
durante l'allenamento.

\subsection{La terza metrica}
Come abbiamo visto in precedenza, come terzo oggetto di studio è stata presa in analisi l'accelerazione del bilanciere 
durante l'esecuzione di un esercizio campione.\\
L'esercizio selezionato è la distensione su panca piana, scelto in quanto ci garantiva il maggior grado di controllo 
sull'esecuzione, essendo uno degli esercizi base di ogni allenamento. Ne possiamo vedere uno schema in Figura 1.\\

La popolarità di tale esercizio ci garantiva inoltre buone potenzialità di espansione di questo studio ad un 
ampio campione di tester, nel caso fossimo riusciti a standardizzare il più possibile il metodo di acquisizione dei dati.

\begin{figure}[htb!]\centering
	\includegraphics[width=\linewidth]{panca_piana}
	\caption{distensioni su panca piana}
	\label{fig:panca_piana}
\end{figure}

\begin{figure*}[ht]\centering % Using \begin{figure*} makes the figure take up the entire width of the page
	\includegraphics[width=\linewidth]{panoramica_palestra}
	\caption{Visione panoramica dell'ambiente}
	\label{fig:view}
\end{figure*}

%------------------------------------------------

\section{Gli strumenti utilizzati}

In questa sezione porremo l'attenzione sugli strumenti di indagine utilizzati, sulla loro costruzione, e sulle 
scelte di progetto e implementazione adottate.

\subsection{Allestimento dell'infrastruttura}

La prima necessità è stata dunque quella di creare l'infrastruttura attraverso la quale sarebbe avvenuta la 
comunicazione dei dati.\\
A livello di rete e internet sono stati fatti diversi tentativi, prima con due diversi router extender che 
prolungavano una rete casalinga fino al locale della palestra, e poi con un router con scheda SIM 
situato direttamente in loco. La prima soluzione è stata scartata in quanto, ad una distanza in linea d'aria 
dal router di casa di circa cinquanta metri, senza considerare i muri, la connessione era troppo instabile.\\

A questa rete è stato poi collegato un Raspberry Pi 4 che ha fatto sia da broker MQTT (di cui parleremo in seguito), 
che da semplice computer per la raccolta e visualizzazione dei dati, come vediamo in Figura 3.

\begin{figure}[htb!]\centering
	\includegraphics[scale=0.32]{schermo_dati}
	\caption{Raspberry e monitor}
	\label{fig:schermo}
\end{figure}

\subsection{I sensori utilizzati}

Per raccogliere i dati sono stati utilizzati i seguenti sensori:

\begin{itemize}[noitemsep] % [noitemsep] removes whitespace between the items for a compact look
	\item due DHT11 e due DHT22 per temperatura e umidità
	\item un CCS811 per eCO\textsubscript{2} e eTVOC
	\item due GY-521 MPU-6050 per l'accelerazione
\end{itemize}

Si è scelto di collegare gli accelerometri a dei moduli di sviluppo ESP32 mini, potendo così accedere ad 
un'architettura double core, mentre tutti gli altri sensori sono stati collegati a delle normali ESP8266. 
Abbiamo adottato sensori DHT11 e DHT22 per temperatura e umidità, potendo così avere un confronto nella 
precisione dei due: vediamo le differenze tra le loro specifiche in Tabella 1.

\begin{table}[hbt]
	\caption{Specifiche DHT}
	\centering
	\begin{tabular}{lcc}
		\toprule
			 & \textbf{DHT11} & \textbf{DHT22} \\
		\midrule
		Range umidità & 20-90\% & 0-100\% \\
		Precisione umidità & ±5\% & ±2\% \\
		Range temperatura & 0-50 °C & -40 +80 °C \\
		Precisione temperatura & ±2\% & ±0.5\% \\
		Tempo di lettura & 6-10 s & 2 s \\
		\bottomrule
	\end{tabular}
	\label{tab:label}
\end{table}

Nelle tre immagini che seguono (Figure 4, 5, 6) sono mostrati gli schemi dei circuiti costruiti: 

\begin{figure}[htb!]\centering
	\includegraphics[scale=0.5]{DHT}
	\caption{ESP8266 - DHT11/22}
	\label{fig:DHT11/22}
\end{figure}

\begin{figure}[htb!]\centering
	\includegraphics[scale=0.4]{CCS811}
	\caption{ESP8266 - CCS811}
	\label{fig:CCS811}
\end{figure}

\begin{figure}[htb!]\centering
	\includegraphics[scale=0.45]{Schema_accelerometro}
	\caption{ESP32 - MPU-6050}
	\label{fig:MPU-6050}
\end{figure}

\subsubsection{I sensori per i dati ambientali e il loro obiettivo}
Tenendo a mente il primo punto di indagine indicato nell'introduzione, ovvero analizzare i dati ambientali, è doveroso 
innanzitutto rispondere al perché abbiamo ritenuto delle metriche tuttosommato banali, 
quali umidità e temperatura, interessanti ai fini dell'elaborato.\\
Come già detto, il locale non presenta sistemi di riscaldamento centralizzato, e, raggiungendo nei mesi invernali 
temperature interne prossime agli zero gradi centigradi, si è scelto di riscaldarlo tramite un generatore di calore a 
GPL con una potenza di 10 KW, posizionato all'ingresso della palestra, ovvero con la parte frontale dentro il locale, 
e lo scarico fuori (si faccia riferimento alla Figura 7).\\

\begin{figure}[htb!]\centering
	\includegraphics[width=\linewidth]{termoventilatore}
	\caption{Termoventilatore GPL}
	\label{fig:termoventilatore}
\end{figure}

Tuttavia questo presentava altri problemi, sia a livello di sicurezza, trattandosi pur sempre di una combustione 
in un ambiente chiuso, per quanto la zona di scarico fosse sufficientemente grande e areata, che a livello pratico:
il generatore di calore utilizzato è in grado di portare l'ambiente ad una temperatura confortevole in tempi 
contenuti, dell'ordine di dieci/quindici minuti (chiaramente in base alla temperatura di partenza), è dunque 
necessario spegnerlo per non surriscaldare l'ambiente. Questo comporta però un crollo della temperatura pressoché 
immediato. Si è scelto dunque di inserire una stufa da 3 KW il cui compito è quello di mantenere la temperatura costante.\\
Alla luce di quanto detto, assume ancora più importanza la seconda metrica oggetto di studio, ovvero la 
variazione di eCO\textsubscript{2} e eTVOC (equivalent Total Volatile Organic Compounds), come monitor indiretto della 
qualità dell'aria.\\
Importante ricordare a tal proposito che il sensore CCS811 non misura direttamente l'anidride carbonica o i composti organici 
volatili, ma misura la quantità equivalente di impurità nell'aria per ottenere quelle letture del sensore: in altre parole, 
il sensore legge determinati dati di qualità dell'aria, tali dati possono essere dovuti alla presenza di CO\textsubscript{2}, 
o da composti organici equivalenti capaci di dare quel tipo di letture. È una sottigliezza ininfluente ai fini dell'elaborato, 
ma per dovere di chiarezza ne abbiamo ritenuto necessaria la precisazione.\\

Data questa premessa passiamo ora a descrivere le relative scelte di progetto.\\
Fare uno studio accurato delle performance termiche di un ambiente richiede una grande quantità di dati, sia a livello 
di caratteristiche ingegneristiche e architettoniche dell'ambiente, che dei materiali di costruzione, informazioni 
alle quali non avevamo completo accesso. Potevamo tuttavia estrapolarle e farne una buona stima, sia attraverso 
l'analisi in loco di periti tecnici, che dai dati restituiti dai sensori.\\

Il dato che si è scelto di ricercare è la dispersione del calore dall'ambiente, che costituisce un buon aggregato e 
indice delle performance termiche della stanza.\\
Per ottenere tale informazione si è scelto di utilizzare i quattro sensori DHT come segue: due sono stati 
posizionati a due angoli della stanza, per ottenere la temperatura media interna, uno fuori dall'edificio per la 
temperatura esterna, e uno appena fuori dall'ingresso della palestra, comunque in un ambiente interno, ma non riscaldato.
\\
Le due postazioni di rilevamento di temperatura e umidità esterne alla stanza, non avendo accesso a prese elettriche, 
sono state implementate con batteria e un con meccanismo di \textit{deep sleep} per il risparmio energetico. 
I relativi circuiti sono stati quindi dotati di collegamenti per una batteria da 650 mA e di un jumper per consentire la 
programmazione del modulo.

\subsubsection{Gli accelerometri e il loro obiettivo}

Un discorso a parte va fatto per gli accelerometri, che per noi costituivano l'argomento di maggiore interesse 
di questo studio.\\
Innanzitutto è necessario chiarire l'obiettivo dell'analisi, ovvero cercare di registrare 
se nell'esecuzione di un qualsiasi esercizio che coinvolge il sollevamento di un bilanciere, si rilevano delle 
imperfezioni, da cogliere attraverso le differenze nelle accelerazioni agli estremi dell'attrezzo: se un accelerometro 
posto a un estremo del bilanciere, avesse riscontrato accelerazioni diverse da quelle registrate 
all'estremo opposto, sarebbe stato possibile individuare degli errori nell'esecuzione dell'esercizio, 
dandoci così un feedback sulla qualità dello stesso, ed eventuali indicazioni per migliorarlo.\\

Con questo obiettivo in mente abbiamo ritenuto imprescindibile che entrambi i moduli fossero alimentati a batteria, 
per evitare perturbazioni nelle misurazioni dovute ad eventuali interferenze dei fili di alimentazione. Questo ha però 
aumentato la difficoltà nella costruzione del \lq\lq sistema accelerometro", che includendo ora anche una batteria, 
diventava più ingombrante e difficilmente stabilizzabile sul bilanciere.\\

La soluzione che abbiamo adottato è stata quella di progettare delle scatole di plastica, appositamente 
disegnate in base alle misure del modulo ESP32 col sensore e la batteria, stampate in 3D. Per minimizzare le vibrazioni 
e i movimenti interni alla scatola, sono stati poi inseriti degli spessori in gommapiuma e dei tappi in plastica, 
aumentando così la stabilità del sistema. Ogni scatola presenta inoltre un foro per accedere alla porta USB dell'ESP32 
per le future programmazioni del modulo.\\
Ne possiamo vedere il risultato nella Figura 8.

\begin{figure}[htb!]\centering
	\includegraphics[width=\linewidth]{scatole_accelerometri}
	\caption{Scatola con accelerometro}
	\label{fig:modulo accelerometro}
\end{figure}

Queste scatole sono state poi fissate agli anelli di bloccaggio dei dischi al bilanciere, come vediamo in Figura 9.

\begin{figure}[htb!]\centering
	\includegraphics[scale=0.15]{disco_bilanciere}
	\caption{Anello di bloccaggio dischi su bilanciere}
	\label{fig:anello di bloccaggio}
\end{figure}

Aggiungiamo inoltre che i bilancieri utilizzati per questi test sono bilancieri standard olimpici, di cui riportiamo 
le misure nell'immagine che segue (Figura 10), che hanno, e questo vedremo in seguito che risulterà critico, 
la caratteristica di avere la zona di fissaggio dei pesi capace di ruotare attorno all'asse del bilanciere. \\
Tale accorgimento tecnico si utilizza per minimizzare la rotazione, e le conseguenti accelerazione e variazione 
di momento angolare del sistema, in fase di movimento del carico: un'improvvisa rotazione del bilanciere attorno 
al suo asse orizzontale non comporta così la rotazione dei pesi (o comunque la riduce significativamente), 
che causerebbe maggiore instabilità nell'esercizio.

\begin{figure}[htb!]\centering
	\includegraphics[width=\linewidth]{bilanciere}
	\caption{Misure bilanciere olimpico}
	\label{fig:bilanciere}
\end{figure}

Prima di passare alla raccolta e analisi dei dati aggiungiamo che i microcontrollori sono stati 
programmati tramite Arduino con la sua integrazione per Visual Studio Code. Eventuali particolarità o dettagli specifici di 
implementazione saranno affrontati successivamente. \\
Tutto il codice è comunque consultabile nella repository GitHub \cite{GitHub} indicata in bibliografia.

\subsection{Tecnologie di comunicazione adottate}

I dati sono stati raccolti su un server remoto sul quale è stato installato \textit{influxdb}, un database 
open source per serie temporali, particolarmente adatto ad applicazioni IOT.\\

Dei sette sensori di cui si è discusso, cinque comunicavano tramite il modulo WiFi del microcontrollore collegandosi 
direttamente al server remoto.\\

\begin{figure}[htb!]\centering
	\includegraphics[width=\linewidth]{pianta_palestra}
	\caption{Pianta della palestra. Non in scala}
	\label{fig:pianta palestra}
\end{figure}

Per i due moduli con DHT11 posizionati fuori dalla palestra, quello interno al locale, e quello esterno all'edificio 
(per maggior chiarezza si faccia riferimento alla pianta del locale in Figura 11) si è optato per una soluzione 
differente: essendo alimentati a batteria, abbiamo scelto di utilizzare due tecnologie particolarmente adatte al 
risparmio energetico. Come protocollo di comunicazione si è scelto MQTT (Message Queuing Telemetry Transport), 
particolarmente meno energivoro del classico WiFi, con il Raspberry Pi 4 che funge da broker, e i due sensori da client. \\
Per l'invio dei dati su influxdb è stato poi appositamente configurato su Raspberry un \textit{plugin telegraf}.\\

La seconda tecnologia per il risparmio energetico è più che altro un accorgimento a livello implementativo: invece di 
programmare i sensori con la classica modalità, ovvero in loop continuo, si è scelto di adottare il \textit{deep sleep}, 
ovvero al termine di ogni lettura e invio dati, i moduli vengono letteralmente addormentati, lasciando attivo il Real 
Time Clock (RTC), che si occupa di risvegliare il sistema dopo un intervallo di tempo programmabile.\\
Nel nostro caso abbiamo visto che la durata della batteria, con questo accorgimento, è praticamente decupliata, 
passando da circa cinque ore a oltre due giorni.\\
Nell'immagine che segue (Figura 12) vediamo uno dei due sensori in oggetto.

\begin{figure}[htb!]\centering
	\includegraphics[width=\linewidth]{DHT-MQTT}
	\caption{Sensore DHT11 fuori dall'ingresso della palestra}
	\label{fig:DHT11 MQTT}
\end{figure}

\section{I dati ambientali durante l'allenamento}

Premettiamo che tutti i dati sono disponibili sia in forma aggregata che estensivamente nella repository GitHub indicata 
in bibliografia \cite{GitHub2}, oltre ad essere consultabili direttamente sul server influxdb.\\

Passiamo ora vedere le misurazioni effettuate, relative ad una data campione particolarmente indicativa, in cui è stata utilizzata 
la palestra. Alleghiamo i grafici estratti da influxdb nelle immagini che seguono (Figure 13, 14, 15), e 
condensati nella tabella in Figura 16.\\

\begin{figure}[htb]\centering
	\includegraphics[width=\linewidth]{temperature_2102}
	\caption{Temperatura in gradi Celsius}
	\label{fig:Variazione temperatura}
\end{figure}
\begin{figure}[htb]\centering
	\includegraphics[width=\linewidth]{umidita_2102}
	\caption{Umidità in percentuale}
	\label{fig:Variazione umidità}
\end{figure}
\begin{figure}[htb]\centering
	\includegraphics[width=\linewidth]{co2_2102}
	\caption{eCO\textsubscript{2} in ppm e eTVOC in ppb}
	\label{fig:Variazione CO2}
\end{figure}

Le linee azzurre e viola sono relative ai due sensori interni, quella arancio al sensore fuori dall'ingresso, 
quella magenta al sensore esterno.

\begin{figure}[htb]\centering
	\includegraphics[width=\linewidth]{tabella_dati}
	\captionsetup{width=\linewidth}
	\caption{dati condensati in timestamp significativi}
	\label{fig:tabella dati}
\end{figure}

\subsection{Analisi dei dati ambientali}
Iniziamo questa analisi da una descrizione più approfondita dei grafici precedentemente allegati.\\
Come possiamo vedere in tabella, sono stati indicati dei timestamp significativi: 

\begin{itemize}[noitemsep] % [noitemsep] removes whitespace between the items for a compact look
	\item 18:55, inizio dell'allenamento e accensione di stufa e generatore di calore
	\item 19:08, spegnimento del generatore di calore 
	\item 19:30 e 20:01, due intertempi per mostrare l'andamento durante l'allenamento
	\item 20:27, spegnimento della stufa
	\item 20:37, fine allenamento
\end{itemize}

La variazione più importante in tutti gli indicatori avviene mentre il generatore di calore a GPL è 
acceso; vediamo infatti che le temperature interne hanno una rapidissima crescita, parallela ad un 
crollo dell'umidità. Molto positivo è inoltre il fatto che la qualità dell'aria non degradi troppo 
significativamente, registrando solo un picco nella quantità di anidride carbonica (e/o suoi composti 
equivalenti), comunque inferiore a qualla generata dal metabolismo durante l'allenamento.\\
Particolarmente interessante risulta l'andamento della temperatura registrata dal sensore posizionato 
appena fuori dall'ingresso della stanza (linea arancio del grafico in Figura 13), che in questa fase 
presenta a sua volta un aumento, mostrando chiaramente i problemi di isolamento
termico dovuti alla separazione degli ambienti ottenuta soltanto tramite due tende in tessuto sintetico.\\
Il discostamento tra i valori della temperatura interna è invece spiegato dall'orientamento della stufa 
elettrica, che evidentemente genera un flusso di aria calda più indirizzato verso uno dei sensori.\\

Dallo spegnimento del generatore GPL in poi, risulta evidente la discesa della temperatura della 
palestra fino ad un punto di stabilità, ovvero una situazione in cui il calore generato dalla stufa 
elettrica e dal metabolismo di chi si allena, è eguagliato dal calore disperso.\\
I due sensori esterni alla palestra mostrano un andamento banale, ovvero di graduale stabilizzazione 
attorno alla temperatura atmosferica per il sensore esterno, e alla temperatura dell'edificio per quello 
fuori dall'ingresso. Per questi ultimi sensori non è di interesse, ai fini dell'elaborato, l'andamento 
dell'umidità.
\\
Torna invece ad essere di particolare interesse notare come le curve di CO\textsubscript{2} e 
umidità abbiano un andamento correlato: i loro aumenti sono dovuti al metabolismo di chi si allena, 
che emette costantemente anidride carbonica, vapore acqueo e composti organici di varia natura.\\

Nella fase terminale dell'allenamento registriamo altre due osservazioni: lo spegnimento della 
stufa causa l'ovvio abbassamento della temperatura interna e la sua convergenza verso un unico valore. 
Si può inoltre notare un difficilmente spiegabile aumento sia dell'umidità che di CO\textsubscript{2} e TVOC.
Questi riteniamo invece che siano dovuti a due fattori: da un lato la natura dell'allenamento svolto, 
che prevedeva una forte componente aerobica sul finale, comportava un forte aumento del metabolismo di chi 
si allenava, dall'altro è possibile che la stufa elettrica fungesse dal moderatore per l'umidità.\\

\section{Studio delle accelerazioni}

I dati sono stati acquisiti durante l'esecuzione di un allenamento così strutturato: ogni esercizio su panca piana 
è stato svolto in blocchi di quattro serie da otto/dieci ripetizioni ognuna, con circa 90 
secondi di pausa tra una serie e l'altra (col termine \lq\lq ripetizione" si intende una discesa e risalita complete 
del bilanciere).\\

\noindent
Gli accelerometri e i microcontrollori utilizzati permettono una lettura dei dati ogni millesimo 
di secondo, ma si è scelto, per non stressare troppo il modulo (causandone surriscaldamento, e relativo degradarsi della 
precisione), di effettuarne una ogni centesimo di secondo. Dilatare ulteriormente la finestra temporale è stato 
tuttavia scartato perché una distensione su panca piana, nelle sue fasi di concentrica e eccentrica (salita e 
discesa del bilanciere), richiede tra gli uno e i due secondi, di conseguenza abbiamo ritenuto opportuno 
avere circa 150 letture per ogni esecuzione.\\
L'implementazione più ovvia, ovvero quella che scrive su database ad ogni lettura (ogni 
centesimo di secondo) è stata scartata per motivi tecnici, ritenedo il tempo di invio dato di gran lunga superiore 
a quello di acquisizione dello stesso: inviare la lettura via internet e scriverla sul server remoto, richiede 
sicuramente molto più di un centesimo di secondo, oltre ad essere un tempo variabile, senza contare la quasi certa 
perdita di pacchetti nel caso di invii così frequenti.\\
Abbiamo dunque aggregato le letture nei loro valori medio e massimo, a partire dagli array sopra descritti, 
la cui dimensione è stata impostata a 35 elementi (corrispondenti ad un invio ogni 350 millesimi di secondo).\\

Si è scelto inoltre di parallelizzare i task principali eseguiti dai moduli, sfruttando l'architettura dual core delle ESP32. 
Invece di lavorare con il classico \textit{while-loop}, vengono associati due task ai due core, eseguiti a loro volta 
su \textit{cicli-for} infiniti. \\
Uno dei due processori si occupa di raccogliere i dati, scrivendoli su un buffer 
condiviso, l'altro controlla, come da paradigma produttore-consumatore, se ci sono dati pronti per essere elaborati (calcolo 
della media e del massimo), e li invia. Il primo core, una volta riempito il primo array, passa a scrivere i dati su un 
secondo buffer.\\
L'idea di fondo è quella di permettere ad un core di effettuare continuamente le misurazioni, mentre all'altro di provvedere 
ai calcoli sui dati e al successivo invio. Così facendo, ipotizzando l'assenza di problemi di perdita dati in fase di 
comunicazione col server, si va ad annullare il tempo morto al termine del riempimento dell'array,
consentendo una scrittura omogenea e continua sul database.\\

\noindent
Prima di passare all'analisi dei grafici ottenuti, schematizziamo i dettagli generali delle scelte implementative a livello 
di codice, che, al solito, rendiamo disponibile nella repository GitHub, dove è disponibile anche lo storico 
delle precedenti implementazioni, essendo quest'ultima figlia di un lungo processo di \textit{trial and error} volto 
all'ottenimento di un'analisi il più indicativa possibile.

\begin{itemize}[noitemsep] % [noitemsep] removes whitespace between the items for a compact look
	\item all'accensione del sensore seguono dieci secondi di \lq\lq buffer" in cui si permette all'utente di 
	posizionare e stabilizzare il modulo sul bilanciere.
	\item terminata questa fase di attesa entrano in esecuzione i due task paralleli sopra descritti
	\item il primo core si occupa di raccogliere i dati negli array, scartando i valori non plausibili, in quanto è stata 
	riscontrata (molto raramente) un'anomalia sulle letture dei sensori, che a volte registrano dei valori di accelerazione fino a 80 $ m/s^2 $.
	I valori scelti come soglie massime nelle tre direzioni spaziali sono stati impostati, in modulo, non superiore a 17 $ m/s^2 $ lungo l'asse 
	verticale, a 5 $ m/s^2 $ nella direzione parallela al bilanciere, e a 10 $ m/s^2 $ per le accelerazioni orizzontali
	\item una volta pronto un buffer di dati, il secondo core procede con l'elaborazione, calcolando 
	per ogni misurazione prima il vettore risultante tramite il teorema di Pitagora in tre dimensioni, e successivamente 
	la media tra tutti i vettori così ottenuti. Oltre a questo viene estratto il valore massimo
	\item sempre il secondo core invia i dati al server con influxdb	
\end{itemize}

\begin{figure}[htb]\centering
	\includegraphics[width=\linewidth]{acc_2104}
	\caption{esercizio completo}
	\label{fig:zoom accelerazione 2104}
\end{figure}

\noindent
In Figura 17 è mostrato l'andamento delle misurazioni relativo a un allenamento di panca piana completo, come sopra descritto; sull’asse 
delle ascisse è rappresentato il tempo, mentre sulle ordinate abbiamo l’accelerazione in $ m/s^2 $. Si può vedere 
chiaramente come nelle zone \lq\lq disturbate" del grafico siano intervallate da zone di relativa quiete: le prime indicano i momenti di 
esecuzione dell'esercizio, mentre le seconde corrispondono alle pause di riposo.\\
In questa figura si può notare come con il passare del tempo, le curve diventino man mano più sporche e caotiche, come se i dati ottenuti 
subissero un degrado progressivo.

\noindent Ai fini dello studio faremo però riferimento principalmente alle Figure 18 e 19, relative a una sola serie di un esercizio.

\begin{figure}[htb]\centering
	\includegraphics[width=\linewidth]{acc_0905_media}
	\caption{accelerazione media}
	\label{fig:zoom accelerazione 0905}
\end{figure}
\begin{figure}[htb]\centering
	\includegraphics[width=\linewidth]{acc_0905_max}
	\caption{accelerazione massima}
	\label{fig:zoom accelerazione 0905}
\end{figure}

Come è purtroppo evidente, i grafici offrono pochi spunti di analisi. Si registra un andamento delle curve in qualche modo correlabile  
a un moto di salita e discesa del bilanciere, ma i grafici relativi all'accelerometro destro e sinistro sono troppo disomogenei 
visto che siamo in presenza di un movimento costante e controllato del bilanciere, che dovrebbe essere testimoniato da due curve pressoché 
sovrapposte, o al limite di poco scostate lungo l'asse del tempo.

\noindent
Una riflessione più approfondita su questi risultati verrà riportata nella sezione successiva.

\section{Risultati dello studio}

\subsection{Risultati delle letture dei dati ambientali}
Con la sola lettura dei grafici è già possibile giungere a conclusioni abbastanza accurate rispetto alle performance termiche 
dell'ambiente, grazie in particolare ad un fattore: nella fase centrale dell'allenamento, avendo come uniche fonti di calore 
quello generato da chi si allena (stimato in letteratura a 200 Watt, per attività di medio-alta intesità) e dalla stufa 
elettrica, possiamo posizionare la dispersione del calore nell'ordine dei 3000 Watt. \\
Con un'analisi più approfondita, e grazie ai dati raccolti dai sensori, siamo stati tuttavia in grado di indagare meglio i 
singoli fattori che contribuiscono a tale dispersione. Il procedimento dettagliato è fornito al solito nei documenti allegati \cite{ing_termo}, 
qui ne mostriamo una sintesi e le conclusioni a cui siamo giunti.\\

In ultima analisi quello che dobbiamo cercare è condensato nella formula $ P = H \Delta T $, dove $ P $ è la potenza dispersa 
media, $ H $ è il coefficiente di scambio termico per trasmissione, e $ \Delta T $ è la differenza di temperatura con l'ambiente 
verso cui avviene la dispersione. L'ultimo termine ci è fornito dai sensori, mentre $ H $ si ottiene da $ H = U A $, 
con $ U $ che è la trasmittanza termica per unità di superficie, valore dipendente dal materiale e noto in letteratura 
\cite{trasmittanza}, e $ A $ è l'area di dispersione.

In Tabella 2 vediamo i valori ottenuti per l'involucro della palestra (per involucro si intende ciò che delimita il volume di un 
ambiente, perciò pareti, pavimentazione, finestre, ecc...).

\begin{table}[hbt]
	\caption{Calcolo del coefficiente di scambio termico}
	\centering
	\begin{tabular}{lccc}
		\toprule
			 & \textbf{U} $ (W/m^2K) $ & \textbf{A} $ (m^2) $ & \textbf{H} $ (W/K) $ \\
		\midrule
		Finestre a un vetro & 6 & 5,73 & 34,38 \\
		Pareti & 1,4 & 68,16 & 95,45 \\
		Soffito e pavimento & 1,65 & 50 & 82,5 \\
		Porta & 20 & 30,31 & 66,2 \\
		\bottomrule
	\end{tabular}
	\label{tab:label}
\end{table}

Sommando i coefficienti così calcolati al coefficiente di scambio termico dovuto alla ventilazione (calcolato in 40,19 $ W/K $), 
otteniamo per $ H $ un valore di 318,69 $ W/K $.
Una prima conclusione che possiamo trarre è che l'isolamento della stanza dal resto dell'edificio, ottenuto con le due tende 
fissate con il velcro, comporta sì una grande dispersione di calore, ma comunque molto minore rispetto a quanto ne viene 
assorbito dagli elementi opachi dell'involucro (pareti, soffitto e pavimento). Questo assorbimento di calore è dovuto alla natura 
di tali elementi opachi: pessimi conduttori e con grande inerzia termica.\\

In Figura 20 possiamo infine vedere le istantanee in momenti significativi di diversi allenamenti, con associate le relative 
temperature e la potenza dispersa in base alla formula che ripetiamo $ P = H \Delta T $. Sono stati evidenziati in verde gli orari 
di accensione del generatore di calore e della stufa elettrica, in arancio e rosso rispettivamente i momenti di spegnimento del 
generatore e della stufa.

\begin{figure}[htb]\centering
	\includegraphics[width=\linewidth]{potenza_dispersa_media}
	\caption{Potenza dispersa in Watt}
	\label{fig:potenza media dispersa}
\end{figure}

In Figura 21 riportiamo uno sguardo approndito sul singolo allenamento del 16 febbraio, giornata con una temperatura esterna 
particolarmente rigida, con l'andamento della potenza dispersa nel tempo.

\begin{figure}[htb]\centering
	\includegraphics[width=\linewidth]{potenza_dispersa}
	\caption{Potenza dispersa durante un allenamento}
	\label{fig:potenza dispersa allenamento}
\end{figure}

La proporzionalità diretta tra la potenza dispersa e l'escursione termica tra la stanza e l'esterno, testimonia come sia 
corretto spegnere il generatore GPL dopo pochi minuti dalla sua accensione, in quanto la percentuale di calore disperso diventa 
sempre maggiore con l'aumento della temperatura interna.

\subsection{Risultati delle letture degli accelerometri}
Nel paragrafo 4 si evince come purtroppo i dati ottenuti dai due moduli con gli accelerometri non consentano nessun tipo di 
analisi in termini di esecuzione dell'esercizio.\\
Il lungo processo di continue modifiche al codice e di vero e proprio \textit{trial and error}, ci ha consentito tuttavia 
di individuare con chiarezza i punti sui quali intervenire.\\

Il primo problema da risolvere è quello delle rotazioni del bilanciere: per quanto i bilancieri olimpionici siano costruiti per 
essere il più giroscopicamente stabili possibile, delle rotazioni possono comunque avvenire, andando a modificare il valore 
dell'accelerazione misurata.\\
Essendo tuttavia un problema di natura prettamente geometrica, si può risolvere essenzialmente in due modi: o cambiando la 
progettazione delle scatole, facendo sì che l'accelerometro si trovi il più possibile vicino al centro 
di rotazione (se non addirittura lungo esso), o intervenendo a livello di codice, catturando con il modulo MPU-6050 non più 
solamente le accelerazioni lungo le tre dimensioni spaziali, ma anche le rotazioni, potendo così perfezionare il dato registrato 
con semplici correzioni matematiche.\\

In seconda istanza sarà da affrontarsi il problema del degradarsi della qualità delle letture nel tempo, ben rappresentato in Figura 17. 
L'indagine in tal senso è più ostica: un fattore che ingenuamente potrebbe esserne causa è il surriscaldamento del modulo ESP32, 
che man mano risulta meno affidabile negli invii dei dati tramite WiFi, visto che si trova compresso tra la plastica della scatola, il sensore, 
la batteria, e un supporto di gommapiuma. Riteniamo poco plausibile che sia questa la causa, ma è bene indagare se ci sia correlazione tra tale 
degrado e l'aumento di temperatura del sensore, e anche in questo caso ci viene in aiuto l'MPU-6050, che essendo dotato anche di un 
termometro interno, può fornirci anche tale informazione.\\

Si faccia infine riferimento alla Figura 22, in cui è raffigurato il grafico delle accelerazioni registrate ai 
capi del bilanciere a seguito di una sollecitazione casuale.	
\begin{figure}[htb]
	\includegraphics[width=\linewidth, height=3cm]{acc_0905_TEST}
	\caption{Scostamento temporale}
	\label{fig:scostamento temporale}
\end{figure}

Quello che possiamo dedurne è di fatto il problema principale di questa analisi: 
tra i dati inviati dai due moduli si registra uno scostamento temporale, per di più casuale, che non dovrebbe essere presente. 
L'analisi accurata delle esportazioni dal database influxdb, che nuovamente lasciamo in allegato, lo testimoniano chiaramente.\\
Fortunatamente è possibile affrontare tale problematica con varie tecniche, delle quali segnaliamo l'assegnazione del 
\textit{timestamp} ai valori delle accelerazioni non più dal server remoto, ma direttamente dai microcontrollori. 
Le ESP32 tuttavia non hanno un orologio interno (oltre al RTC, che però è inutile ai fini della sincronizzazione), sarà dunque 
necessario implementare un meccanismo per ottenere, e soprattutto mantenere, la sincronia tra le letture dei due moduli.

\section{Sviluppi futuri}
Riteniamo lo studio dei dati ambientali soddisfacente e esaustivo, per quanto sarebbe di indubbio interesse  
approfondire lo studio dei composti organici nell'aria in relazione al metabolismo del soggetto.\\

È tuttavia agli sviluppi degli accelerometri che intendiamo dedicare le ricerche future, in quanto vediamo in questi delle 
buone potenzialità, sia a livello commerciale, che di ricerca ai fini della biomeccanica del movimento, branca delle scienze motorie 
sempre più coadiuvata dalle moderne tecnologie. Segnaliamo in particolare la possibilità
di includere un modello di \textit{Machine learning per l'anomaly detection} nell'esecuzione dell'esercizio.\\
Inoltre, una volta ottenuto un prototipo funzionante nel caso specifico delle distensioni su panca piana, questo sarà adattabile ad ogni 
esercizio che coinvolga il sollevamento di un bilanciere.

%----------------------------------------------------------------------------------------
%	REFERENCE LIST
%----------------------------------------------------------------------------------------

% \phantomsection
% \bibliographystyle{unsrt}
% \bibliography{report.bib}

\vspace{5cm}
\printbibliography

%----------------------------------------------------------------------------------------

\end{document}